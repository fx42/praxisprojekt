%----------------- KONFIGURATION ----------------- %
\thispagestyle{empty}

\section*{Kurzfassung}
\label{cha:kurzfassung}
Im Zuge meiner praktischen Studienphase wurde ich bei der Bosch Software Innovations GmbH mit vielen Technologien, technischen Ansätzen und Konventionen in der modernen Softwareentwicklung konfrontiert. Durch diese Erfahrungen entstand der folgende Artikel über den REST Architekturstil. Behandelt wird die Entstehung sowie die grundlegenden Eigenschaften die eine Applikation mit sich bringen muss, um als RESTful-API zu funktionieren. Bevor REST das Feld bei der Entwicklung von Web-Applikation übernommen hat, war die Verwendung von SOAP als Standardvorgehen unabdingbar. Im Artikel findet sich eine Gegenüberstellung sowie ein entsprechendes Fazit. Anhand eines Beispiels wird unter Verwendung des Jersey 2.0 Frameworks eine Implementierung einer REST-Schnittstelle in Java beschrieben und entscheidende Eigenschaften werden erläutert.

\vspace{5em}
\renewcommand{\cleardoublepage}{}
\renewcommand{\clearpage}{}
\section*{Abstract}\label{cha:abtract}
In the course of my internship at Bosch Software Innovations GmbH I was confronted with a lot of technologies, technical approaches and conventions of the modern software development. Through these experiences the following article about the REST architecture style aroses. The article starts with an explanation of the origin of this architecture style and which constraints an application needs to fulfill to  perform as a RESTful-API. Before REST tooks place in the Web-Application development SOAP was the common representative for this kind of development. A confrontation between REST and SOAP with the resultant conclusion takes place in this article also. With the usage of the Jersey 2.0 Java Framework an example gets implemented and the necessary characteristics are explained. 
\vspace{10em}
