%----------------- KONFIGURATION ----------------- %
\pagestyle{empty} % enthalten keinerlei Kopf oder Fuß

\section*{Kurzfassung}
\label{cha:kurzfassung}
Im Zuge meiner praktischen Studienphase wurde ich bei der Bosch Software Innovations GmbH mit vielen Technologien, technischen Ansätzen und Konventionen in der modernen Softwareentwicklung konfrontiert. Durch diese Erfahrungen entstand der folgende Artikel über den Architekturstil REST. Behandelt wird Entstehung sowie die grundlegenden Eigenschaften die eine Applikation mit sich bringen muss, um als REST-API zu funktionieren. Bevor REST das Feld bei der Entwicklung von Web-Applikation übernommen hat, war die Verwendung von SOAP als Standardvorgehen unabdingbar. Im Artikel findet sich eine Gegenüberstellung sowie eine Evaluierung der Unterschiede. Anhand eines Beispiels wird unter Verwendung des Jersey 2.0 Frameworks eine Implementierung einer REST-Schnittstelle in Java beschrieben und entscheidende Eigenschaften werden erläutert.

\vspace{5em}
\renewcommand{\cleardoublepage}{}
\renewcommand{\clearpage}{}
\section*{Abstract}\label{cha:abtract}
In the course of my internship at Bosch Software Innovations GmbH I was confronted with a lot of technologies, technical approaches and conventions of the modern software development. Through these experiences the following article about the REST architecture style aroses. The article starts with an explanation of the origin of this architecture style and which constraints an application needs to fulfill to  perform as a REST-API. 

This report is based on the experiences of working on various project.It outlines the set up and project management as well as the technology that was used and focuses on a selection of performed tasks. These refers to both typical and specific tasks of a web developer. The described development work is mostly founded on JavaScript and CSS.
\vspace{5em}
